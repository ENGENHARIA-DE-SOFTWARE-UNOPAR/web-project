
\section{Introdução}

\par A programação web vem ganhando cada vez mais espaço na mundo globalizado, nos dias atuais torne-se praticamente inviavel a instituições dos mais variados tipos, que não possuem um web sitio e neste aspecto que secapacita profissionais para este leque.
\par Segundo \citeonline{alvaron:2024}, empreas que não possuem site perdem clientes para concorrentes, pois sem um site os sites de pesquisas não irão encontrar a sua empresa para aquisição de algum ou alguns produtos ou serviços.




\section{Métodos}
\par Para esta aula prática foi proposto um roteiro, está disposto em: \href {https://github.com/ENGENHARIA-DE-SOFTWARE-UNOPAR/web-project/blob/main/Roteiro%20aula%20pr%C3%A1tica.pdf} {roteiro da aula prática}. De igual modo cria-se um repositório no GitHub para o versionamento da referiada aula prática, e que pode ser acessado atráves deste \href {https://github.com/ENGENHARIA-DE-SOFTWARE-UNOPAR/web-project} {link}.
\par Neste vies foi eleito o modelo de relatórios em \textbf{LaTeX} para relatórios, pois o mesmo acaba automatizando alguns aspectos

\section{Resultados}



\lstinputlisting[language=php, caption={código externo}, label={cod:externo}]{main.c} %Busca os codigos na pasta /cod




\begin{figure}[H] %Figuras da aula pratica 1.1
  \center
  \subfigure[ Algoritmo.\label{fig:pri2}]{\includegraphics[scale=0.4]{figure/placeholder.jpg}}
  \subfigure[Comportamento.\label{fig:seg2}]{\includegraphics[scale=.4]{figure/placeholder.jpg}}
  \caption{Resultado da atividade prática 1.2, \cite{oliveira_SO2009}}\label{fig:ap1_cod_vigual1}
\end{figure}

%%%%%%%%%%%%%%%%%%%%%%%%%%%%%%%%%%%%%%%%%%%%%%%%%%%%%%%%%%


Para referenciar utilize \cite{ninguem2022curioso}. Também pode ser citado integrada ao texto, de acordo com \citeonline{alguem2022nada}.


\par Estou usando \href {https://cocalc.com/} {CoCal}

E para referenciar a figura \ref{fig:imagem_massa} utilize dessa forma.


\begin{enumerate}[label=\Roman{*}, ref=(\roman{*})]
  \item fsfsdf
  \item kugfhiuh
\end{enumerate}

\begin{asparaenum}
\item Anterior ... \cite{ninguem2022curioso}
\item Próximo ... \label{pl1}
\end{asparaenum}



\section{Conclusões}




  %$X \xLongleftarrow[\text{NATAN}]{\text{OGLIARI}} Y $ %COM TEXTO
	% $\uparrow$ %Seta para Cima
	%$\overleftarrow{NATAN}$
